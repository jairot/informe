\chapter{Conclusiones}

En primer lugar, puede concluirse que existe factibilidad en la integración de módulos que realizan procesamiento de paquetes por hardware con aquellos que lo llevan a cabo por software.

Como se pudo ver en este trabajo la arquitectura utilizada es claramente modular y por lo tanto propicia para la integración de nuevos módulos. Esto último resulta fundamental por estos días debido a las crecientes necesidades en velocidad y ancho de banda.

Por otra parte, el hecho de no haber llevado a cabo pruebas con una interfaz de red real es compensado con el módulo \textit{ad-hoc} generador de paquetes, el cual brinda la posibilidad de generar tráfico Ethernet a la hora de realizar los testeos.

Se pudieron estudiar diversos algoritmos de clasificación para luego implementar 2 de ellos sobre una lógica reconfigurable y analizar comparativamente su performance.

Es posible efectuar mejoras en los esquemas implementados. Para el caso de una búsqueda lineal, puede implementarse una cola con prioridades para evitar el uso de templates y el overhead que ello implica. El el caso de la búsqueda en árbol puede optarse por un esquema multibit, que evite la expansión excesiva del árbol que contiene los prefijos 

\subsubsection{Trabajos Futuros}

Es posible realizar varias mejoras y extensiones a este proyecto, a continuación se muestra una lista de estas mejoras futuras:

\begin{itemize}
	\item asdsad
\end{itemize}






%\section{Distribucion Lineal}
