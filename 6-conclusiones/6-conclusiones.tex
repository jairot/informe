\chapter{Conclusiones}


En base a los resultados observados en el presente trabajo, se puede concluir que es factible implementar una arquitectura de clasificación de paquetes en lógica reconfigurable, combinando las velocidad del procesamiento de paquetes por hardware con la escalabilidad y estabilidad que permite el software. 

Se considera que la complejidad de este tipo de proyecto reside especialmente en la implementación del SoC y, más específicamente, en la curva de aprendizaje pronunciada que presentan las herramientas utilizadas para desarrollar este tipo de sistemas. En un futuro se espera que las herramientas maduren para así permitir una mayor fluidez y simplicidad en el diseño y puesta en marcha de este tipo de componentes. 

Por otro lado, la integración de módulos propios a un bus pre-determinado depende fuertemente de la calidad de la documentación que este presenta. Es por ello que a la hora de seleccionar un microprocesador para integrar un SoC en lógica reconfigurable, es fundamental revisar la documentación y saber que se cuenta además con una comunidad activa y colaborativa que puede asistir a los diseñadores durante el proceso de implementación. 

Esta implementación es multidiciplinaria, ya que para su realización es necesario contar con conocimientos en el desarrollo de hardware, software y  redes, precisamente los tres ejes de Ingeniería en Computación, la carrera para la cual se desarrolló este Proyecto integrador. 

Aunque se trata de una implementación experimental, la performance del sistema no se encuentra alejada de la que alcanzan los productos que se encuentran presentes en el mercado actual, y combinado con otros desarrollos realizados en el Laboratorio de Comunicaciones Digitales es posible alcanzar un producto acabado y con capacidades de resolver situaciones reales.

Se destaca la arquitectura modular implementada que, junto con el desarrollo en C++ del software, permite portar la misma a otras plataformas.

Es de importancia señalar que se presento una versión reducida de este proyecto en el Congreso de la Red UNIC 2011(CRUNIC2011).

Por ultimo, entendemos durante la realización de este proyecto pudimos aplicar los conceptos aprendidos durante la carrera de Ingeniería en Computación, así  desarrollar nuevas competencias. 



\subsubsection{Trabajos Futuros}

Es posible realizar varias mejoras y extensiones a este proyecto, a continuación se muestra una lista de posibles propuestas:

\begin{itemize}
	\item Implementar algoritmos de clasificación multidimensionales y con características más complejas.
	\item Optimizar el código de los algoritmos para el procesador Nios II. 
	\item Portar este sistema a otras plataformas.
	\item Implementar una cache por hardware.
	\item Implementar mecanismos para modificar y actualizar de manera automática la tabla de búsqueda.
	\item Integrar este proyecto con el proyecto integrador "Diseño de Arquitectura Hardware Reconfigurable para la Conmutación de Paquetes en Redes Virtuales Privadas"\cite{spaz}, desarrollado en este laboratorio.
\end{itemize}






%\section{Distribucion Lineal}
